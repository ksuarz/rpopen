\documentclass{article}
,
\usepackage{verbatim}

\begin{document}
\title{Homework 2: Socket Programming}
\author{Kyle Suarez}
\date{\today}
\maketitle

\section{Code}
\subsection{rpopen}
Implements a remote version of \texttt{popen(2)}. On calling \texttt{rpopen},
this simply creates a new socket, sets up a connection with an rpserver and
communicates with it to execute an arbitrary command remotely.

\subsection{rpserver}
\texttt{rpserver} is a daemon that listens for incoming connections on a socket.
After receiving a connection on the socket from a client calling
\texttt{rpopen}, the server forks and has its child execute a specified command.

\section{Compiling and Running}
To compile everything, simply use the Makefile:

\texttt{\$ make}.

On my home computer, I compiled this with Clang with the following options:

\texttt{\$ cc -g -Wall *.c}

\section{Tests}
I ran the following tests:
\begin{enumerate}
    % TODO
\end{enumerate}
\end{document}
