\documentclass{article}
,
\usepackage{verbatim}

\begin{document}
\title{Homework 2: Socket Programming}
\author{Kyle Suarez}
\date{\today}
\maketitle

\section{Code}
\subsection{rpcommon}
This has two functions (\texttt{rpopen_get_socket} and
\texttt{rpopen_bind_socket}) for handling sockets that are common to both the
server and the client.

\subsection{rpopen}
Implements a remote version of \texttt{popen}. On calling \texttt{rpopen},
this simply creates a new socket, sets up a connection with an rpserver and
writes the command it wants to execute over the socket. Then, it obtains a file
descriptor from the socket via \texttt{fdopen} and returns it to the caller.

\subsection{rpserver}
\texttt{rpserver} is a daemon that listens for incoming connections on a socket.
After receiving a connection on the socket from a client calling
\texttt{rpopen}, the server forks and has its child execute a specified command.

\section{Compiling and Running}
To compile everything, simply use the Makefile. I added another recipe for
\texttt{rpcommon.o}.

On my home computer, I compiled this with Clang with the \texttt{-g -Wall}
options.

\section{Tests}
I ran the following tests:
\begin{itemize}
    \item A regular command. This worked as expected.
    \item An invalid command. The client receives nothing; errors are printed to
    standard error on the server side.
    \item A command that required elevated privileges. This failed as expected
    and an error message was printed on the command line.
\end{itemize}
\end{document}
